\chapter{Hjem-Is interview}\label{app:interview}
\begin{enumerate}
    \item Hvad er I for en virksomhed?
    \begin{enumerate}
        Fødevarevirksomhed (frostfødevarer - der er forskellige regler alt efter hvilke fødevarer man håndterer)
    \end{enumerate}
    \item Virksomheden startede med is bag på cyklen, så bilen, og så videre
    \item Hvilke forretningsområder dækker I?
    \begin{enumerate}
        \item At køre ud med is, detailhandel. 
        \item Primært til private, men også til erhverv
        \item Samme produkter til både erhverv og private
        \item Arrangementer kan få en is-bil ud.
        \item Om sommeren er der en del erhvervskunder
    \end{enumerate}
    \item Hvad er jeres målgruppe?
    \begin{enumerate}
        \item Landsdækkende, men kører på alle villaveje
        \item På villavejene er der flest børnefamilier        
    \end{enumerate}
    \item Hvor mange ansatte er I i Aalborg?
    \begin{enumerate}
        \item Om vinteren er de 3 fuldtidsansatte \item ismænd, chefen, og Helle på deltid
        \item Får én til fuldtidsansat
        \item Om sommeren er der 9 fuldtidsansatte
    \end{enumerate}
    \item Hvordan er strukturen i organisationen? Hvor mange lag, og kommunikation imellem
    \begin{enumerate}
        \item I depotet er der kun 2 niveauer, og nem kommunikation mellem ledere og medarbejdere
        \item ift. hovedkontoret er der længere
    \end{enumerate}
    \item Hvilke stillinger er der i depotet?
    \begin{enumerate}
        \item Ismand (3 fuldtidsansatte)
        \item Lageransvarlig 
        \item deltids ismand
        \item Depotleder
    \end{enumerate}
    \item Hvordan ville du beskrive din arbejdsdag?
    \begin{enumerate}
        \item Starter med at tælle penge op fra dagen før
        \item Tæller dagsomsætningen om på hver bil, registrere salg
        \item sætte penge i banken
        \item Skriver ordre ud til alle isbiler
        \item fylder isbilerne op
        \item Laver turer til bilerne
        \item Ugeplan med hvilke medarbejdere
    \end{enumerate}
    \item Hvilke kompetencer er I interesserede i at jeres medarbejdere har?
    \begin{enumerate}
        \item God til at sælge
        \item Godt humør
        \item Lyst til at lave arbejdet
    \end{enumerate}
    \item Hvem er jeres konkurrenter?
    \begin{enumerate}
        \item Netto, fakta, rema, supermarkederne - de er overalt, og man kan få én ad gangen. Det er også billigere
        \item “vi skal eddermame lave nogle is der er specielt gode hvis vi skal blive her [på markedet], og det skal vi”
    \end{enumerate}
    \item Hvordan er I forskellige fra andre virksomheder/konkurrenter
    \begin{enumerate}
        \item Hjem-IS leverer ud til kundens adresse
        \item Sælger i pakker i stedet for enkelte is
        \item “vi er stort set den sidste specialforretning i Danmark”
    \end{enumerate}
    \item Hvad for et system bruger I til regnskab osv.?
    \begin{enumerate}
        \item VISMA - Løn 
        \item Til bogføring bruges et andet system
        \item Systemet blev opdateret konstant fordi det ikke virkede i starten
        \item Virker nu, så færre opdateringer
    \end{enumerate}
    \item Er lønnen provisionslønnet?
    \begin{enumerate}
        \item 152kr i timen.
        \item Der er en kvote, men de får også 10\% af al salg udover kvoten
        \item Der er ikke mange vagter om vinteren, men en del flere om sommeren
    \end{enumerate}
    \item Hvordan får i isbiler hjem?
    \begin{enumerate}
        \item De får dem hos Ford, som spænder is-kasse på
        \item Det er hovedkontoret der styrer det
    \end{enumerate}
    \item Har i andre leverandører?
    \begin{enumerate}
        \item Nej.
        \item Hjem-IS og Premier Is er gået sammen, så varerne kommer fra enten premier eller hjem-is selv.
        \item Hvordan fabrikkerne laver isen og hvilke levenrandører de har fik vi ikke at vide
    \end{enumerate}
    \item Kundeinformationer - Er der bestemte informationer I gemmer?
    \begin{enumerate}
        \item Gemmer kvitteringer i 5 år, samt kundeoplysninger nødvendigt for at kunne gennemføre købet
        \item Bruger e-pay til online betalinger
        \item Man kan melde sig ind i Hjem-IS klubben og så kan Hjem-IS se hvad de køber, hvor meget og hvor
    \end{enumerate}
    \item Hvad bruger I til ruteplanlægning?
    \begin{enumerate}
        \item Google maps integration
    \end{enumerate}
    \item Hvilke udfordringer har I med systemet i dag?
    \begin{enumerate}
        \item Når man skal slette stop på ruten, skal man markére alle stop individuelt i stedet for at man kan vælge mange på én gang og slette
        \item Kalenderen samarbejder ikke med optællingssystemet - Det kunne være smart hvis de var forbundet i stedet for at man manuelt skal indtaste data fra turerne i kalenderen og regnskabet (excel)
    \end{enumerate}
    \item Hvad er der på dashboardet?
    \begin{enumerate}
        \item Alt der har med salg at gøre
        \item Budget for dagen, tur-snit, kunde-køb gennemsnit
        \item Hvordan aalborg depotet ligger ift. de andre depoter
        \item Mulighed for at se hvad hver enkelt sælger har solgt for og hvordan de ligger ift. budget
        \item Man kan se hvilke varer der er solgt, hvor mange der er solgt, rangliste over varerne
        \item SKU - hver is’ ID
    \end{enumerate}
    \item Er der nogle bestemte ting der skulle laves i et nyt system?
    \begin{enumerate}
        \item Se udfordringer (punkt 18)
        \item Samarbejde mellem kalender, salgstal og overførsel i Excel
    \end{enumerate}
    \item Hvad bruger I til lagerstyring?
    \begin{enumerate}
        \item tja
        \item Der er automatisk bestilling af varer der mangler
        \item Minimumsmængden er bestemt manuelt, det er ikke beregnet ud fra hvor mange der bliver solgt eller prisen
        \item Kræver at hvert salg noteres korrekt - Det kan se at der bliver indtastet den forkerte is ved salget, selvom det er samme pris. det fucker lageret lidt op. Pakkerne bliver ikke scannet, men tastet ind manuelt på en iPad
    \end{enumerate}
    \item Hvordan korrigerer i sådan en fejl med forkert vare?
    \begin{enumerate}
        \item Det skal siges hvis fejlen er lavet, ellers går det ubemærket frem til lageroptælling
    \end{enumerate}
    \item Er der nogle is der er mere eftertragtede end andre?
    \begin{enumerate}
        \item Jep - Det skal rettes til manuelt så lageret ikke kommer i underskud af de populære is.
        \item Lad os lave noget machine learning som kan fikse det her ved at rette minimumsantal ud fra om det er på tilbud, om der bliver solgt mange eller få, og om isen er ved at udgå
    \end{enumerate}
\end{enumerate}

\textbf{Ekstra:}
\begin{itemize}
    \item De kører alle steder men ikke så ofte om vinteren
    \item Til Marts og om sommeren kører de mindst en gang om ugen, og primært i villakvarterer, men også ude ved sommerhus områder
    \item Det er en presset virksomhed - for 25 år siden var de kongerne af vejen. Nu skal Hjem-IS til at fedte for kunderne. For 10 år siden solgte de for næsten det dobbelte, og det ligner ikke at det kan vendes om.
    \item Udfordringerne som forårsager at omsætningen falder med årene er, at Hjem-IS er meget detail orienteret, men bliver nemt udkonkurreret af supermarkederne.
    \item Niels (chefen) vil gerne bruge mindre tid på administration, og mere tid på at køre ud og sælge
\end{itemize}