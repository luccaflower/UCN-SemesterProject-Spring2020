\ucntitlepage{%
  \danishprojectinfo{
    Automatisk lagerstyring %title
  }{%
    Forårssemester 2020 %project period
  }{%
    dmab0919 - gruppe 4 % project group
  }{%
    %list of group members
    Benyad Jomhur\\ 
    Lucca Christiansen\\
    Mads Østermark\\
    Mathias Olesen\\
    Søren Ravn\\
  }{%
    %list of supervisors
    Istvan Knoll\\
    Lars Nysom\\
    Per Trosborg\\
    Torben Larsen    
  }{%
    87.000 %\todo{update this number}%anslag
  }{%
    xx %normalsider = anslag/2400
  }{%
    \url{https://github.com/cramt/hjem-is}, commit nr. 129, c0a7600 %\todo{do we include the latest commit as version number?}
  }{%
    \today % date of completion
  }%
}{%department and address
  \textbf{IT-uddannelserne}\\
  Professionshøjskolen UCN\\
  \href{http://www.ucn.dk}{www.ucn.dk}
}{% the abstract
  This report focuses on developing a solution for a problem Hjem-IS has. From an interview with the company, the purpose of the paper is to show the development of this system, and show which steps are being taken, how and why. Using Unified Process, first a business analysis was made to find the problem to solve, which was unoptimal time usage. A solution to this problem was abstracted by using Use Cases, from which the requirements and design of the system was made, so finally the system could be coded and tested in Java. The paper concludes with providing a proof of concept of this system, with a GUI and tests with working regression methods. Had there been more time for multiple iterations during the Unified Process development, these regression algorithms could be implemented to continuously create optimal orders of products to increase profit and increase available time for other tasks.
}{
  
  
}
