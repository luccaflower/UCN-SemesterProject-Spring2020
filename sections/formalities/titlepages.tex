\ucntitlepage{%
  \danishprojectinfo{
    Automatisk lagerstyring %title
  }{%
    Forårssemester 2020 %project period
  }{%
    dmab0919 - gruppe 4 % project group
  }{%
    %list of group members
    Benyad Jomhur\\ 
    Lucca Christiansen\\
    Mads Østermark\\
    Mathias Olesen\\
    Søren Ravn\\
  }{%
    %list of supervisors
    Istvan Knoll\\
    Lars Nysom\\
    Per Trosborg\\
    Torben Larsen    
  }{%
    xx %anslag
  }{%
    xx %normalsider    
  }{%
    \url{https://github.com/cramt/hjem-is}
  }{%
    \today % date of completion
  }%
}{%department and address
  \textbf{IT-uddannelserne}\\
  Professionshøjskolen UCN\\
  \href{http://www.ucn.dk}{www.ucn.dk}
}{% the abstract
  This report focuses on developing a solution for a company to implement an automatic storage system. The purpose of the paper is to show the development of this system, and show which steps are being taken and how. Using Unified Process, first a business analysis was made to find a problem to solve, which was to create an automatic storage system. A solution was abstracted by using Use Cases and then the requirements and design of the system was made, so finally the system could be coded in Java. The paper concludes with providing a proof of concept of this system with a UI and tests with working regression methods. Had there been more time for multiple iterations during the Unified Process development, these regression algorithms could be implemented to continuously create optimal orders of products to increase profit and lower mistakes made.
}{
  
  
}
