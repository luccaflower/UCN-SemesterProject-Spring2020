\chapter{Design}\label{ch:design}
Nu hvor der både er taget højde for de informationer og funktioner, der skal være i systemet, kan der laves et interaktions- og designklassediagram. De vil blive designet på baggrund af diagrammerne fra Analyse afsnittet og trelagsarktitekturen. 

\section{Kommunikationsdiagram}
Kommunikationsdiagrammet \cite{Larman2004} på Figur \ref{fig:Kommunikationsdiagram} viser de centrale trin fra Fully Dressed Use Casen Automatisk Lagerstyring, som også er vist i SSD på Figur \ref{fig:ssd}. Kommunikationsdiagrammet bruges til at vise hvilke klasser, der laver hvilke metodekald og hvordan klasserne kommunikerer med hinanden. 
Som det fremgår på Kommunikationsdiagrammet Figur \ref{fig:Kommunikationsdiagram}, kan der ses at StoragePLanListUI kan generere en ny lagerplan. Når den nye lagerplan er fremkaldt, kan brugeren navngive planen og åbne den. Herfra kan brugeren tilpasse planen til behovene, dette gælder både under StoragePlanUI og PeriodicPlanUI. 

Der er brugt flere UI paneler, som vinduer og undermenuer, hvori de hver har deres egne funktion. 
F.eks, så har PeriodicPlanUI som funktion at oprette og redigere perioder, samt oprette, tilføje og fjerne produkter. 

Under observation af Kommunikationsdiagrammet, kan det ses, at det meste af arbejdet udføres af StoragePlanController, derfor er den den mest komplekse. Herunder kan det observeres at Sales er med i Kommunikationsdiagrammet. Grunden til dette, er at Sale har indflydelse på, hvor meget der bestilles hjem til lageret, hvor meget der er på lageret og hvilken periode der bliver solgt flere af en bestemt variant. 
Sale blev dog prioriteret væk, fordi der kun var tid til at teste om det kunne virke med eksempel.salgsdata.
Idéelt ville Sale være implementeret, sådan at Sales blev brugt til at lave lagerplanerne ud fra.

\begin{landscape}
    \begin{figure}[p]
        \centering
        \includegraphics[width=0.8\hsize]{figures/design/Kommunikationsdiagram.eps}
        \caption{Kommunikationsdiagram}
        \label{fig:Kommunikationsdiagram}
    \end{figure}
\end{landscape}

\section{Designklassediagram}
Designklassediagrammet er det tætteste vi kommer på en præcis repræsentation af programmet. Den viser samtlige klasser, metoder, attributter og relationer, så det blot er kildekoden der skal implementeres. Designklassediagrammet giver også et godt overblik arkitekturen af programmet, og Designklassediagrammet skulle gerne reflektere domænemodellen, og Kommunikationsdiagrammet.

Som det kan ses på Figur \ref{fig:designclassdiagram} er der brugt en afslappet lagdelt arkitektur, hvori der indgår et UI lag, et controller lag, et model-og-database lag \cite{Larman2004}. UI laget er brugergrænsefladen og tillader brugeren at interagere med programmet. Controller laget håndterer informationsændringer mv. og er de centrale komponenter til at udføre Use Casen. Model-og-database laget indeholder de klasser, der fremgik i domænemodellen i Figur \ref{fig:domain_model_2}, deres attributter og metoder - i database-delen sørges der for persistens og organisering af data. 
For at kunne gennemføre en god lagdelt struktur, kræves der forskellige designmønstre. Det kan ses at DBConnection er baseret på et Singleton mønster \cite{Larman2004}, som sørger for at der kun oprettes én instans af DBConnection. DBConnection benyttes af alle \verb|SqlStore| klasserne for at kunne gemme data i databasen. Udover det ses det også, at der er benyttet et DAO mønster \cite{DAO} med alle Controller klasser. Dette ses ud fra at alle controller-klasser har en interface til database operationer, og ved forholdet til modelklasserne.
I forhold til UI, ses det at Myframe er en superklasse, som vil indeholde de tre UI subklasser, så det bliver nemmere at skifte mellem faner i brugergrænsefladen.
Diagrammet kan oversættes til kode, så arkitekturen, attributter og metoder kan implementeres.

\begin{landscape}
    \begin{figure}
        \centering
        \includegraphics[width=1\hsize]{figures/design/actual_designclass_diagram.eps}
        \caption{Designklassediagram}
        \label{fig:designclassdiagram}
    \end{figure}
\end{landscape}

