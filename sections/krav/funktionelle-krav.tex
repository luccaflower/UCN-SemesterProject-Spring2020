\section{Funktionelle krav}\label{sec:funktionelle-krav}
\section{Fully Dressed}

Herunder ses Fully Dressed Use Casen ‘Automatisk lagerstyring’. Use case 'Automatisk Lagerstyring' er bygget på de tidligere use cases: 'Afskriv vare', fordi den påvirker hvor meget der er på lager. 'Optælling af isbil', som er manuelt arbejde og vil nemt kunne erstattes med automatisk optælling. 'Bestil varer', dette ville gå hånd i hånd med automatisk optælling af bilen. Og til sidst use case 'Modtag varer', fordi systemet skal opdateres med de nye varer der kommer på lager. 

De alternative flows beskrevet i diagrammet er fundet gennem en diskussion, hvor mulige fejl eller forhindringer kan opstå under flowet af handlinger. De alternative flows beskriver, hvordan et system kan fejle undervejs, hvad der er nødvendigt at være forberedt på, og hvordan disse alternative flows kan behandles så systemet ikke stopper, men kan fortsætte Use Casen til ende. Sker det at systemet eller kunden ikke opfører sig som forventet, skal de mulige udfald udtænkes, og sådanne scenarier skal kunne behandles af systemet så vidt muligt.


\begin{longtable}{ |p{120pt}|p{120pt}|p{120pt}| }
    \hline
    Use case navn & Automatisk lagerstyring & \\
    \hline
    Aktør & Depotchefen & \\
    \hline
    Præbetingelser & Et lager med en addresse, penge til at bestille, lagerbeholding fysisk er det samme som i systemet, having a little bit sales data & \\
    \hline
    Postbetingelser & Et lager det er opfylgt med IS & \\
    \hline
    Frekvens & 1 gang om dagen & \\
    \hline
    Main Success Scenario (Flow of events) & Aktørhandling & Systemsvar \\
    \hline
    & 1. vis salgs statistik & 2. en hel masse statistik \\
    \hline
    & 3. klikker "generer lagerplan" & 4. fil dialog omkring hvor planen skal gemmes og er automaitsk sendt til plan-fanen \\
    \hline
    & 5. klikker "load plan" & 6. en eller anden form for UX feedback \\
    \hline
    Alternative flows & 5a. Aktøren vil gerne modificere planen, og gør det i plan-fanen \\
    \hline
\end{longtable}

\begin{longtable}{ |p{120pt}|p{120pt}|p{120pt}| }
    \hline
    Use case navn & Automatisk bogføring & \\
    \hline
    Aktør & Depotchefen & \\
    \hline
    Præbetingelser & ingen & \\
    \hline
    Postbetingelser & bogføring er automatisk færdiggjord i et excel ark & \\
    \hline
    Frekvens & 1 gang om dagen & \\
    \hline
    Main Success Scenario (Flow of events) & Aktørhandling & Systemsvar \\
    \hline
    & 1. klikker "export bogføring" & 2. systemet exporter en excel fil med bogføring \\
    \hline
\end{longtable}