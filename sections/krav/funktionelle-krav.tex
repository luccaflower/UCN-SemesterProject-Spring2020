\section{Funktionelle krav}\label{sec:funktionelle-krav}
Dette afsnit vil behandle de funktionelle krav der skal være til systemet. De funktionelle krav omhandler de krav, der stilles til funktionaliteten af programmet gennem use cases\cite{Larman2004}. 

\subsection{Fully Dressed} \label{fullydressed}
Herunder ses Fully Dressed Use Casen Automatisk lagerstyring. Use case Automatisk Lagerstyring er bygget på de tidligere use cases: Afskriv vare, fordi den påvirker hvor meget der er på lager. Optælling af isbil, som er manuelt arbejde og vil nemt kunne erstattes med automatisk optælling. Bestil varer, ville gå hånd i hånd med automatisk optælling af bilen. Og til sidst use case Modtag varer, fordi systemet skal opdateres med de nye varer der kommer på lager. 

De alternative flows beskrevet i diagrammet er fundet gennem en diskussion, hvor mulige fejl eller forhindringer kan opstå under flowet af handlinger. De alternative flows beskriver, hvordan et system kan fejle undervejs, hvad der er nødvendigt at være forberedt på, og hvordan disse alternative flows kan behandles så systemet ikke stopper, men kan fortsætte Use Casen til ende. Sker det at systemet eller kunden ikke opfører sig som forventet, skal de mulige udfald udtænkes, og sådanne scenarier skal kunne behandles af systemet så vidt muligt.

\todo{update so it's consistent with new ssd}
\begin{longtable}{ |p{120pt}|p{120pt}|p{120pt}| }
    \hline
    \textbf{Use case navn} & Automatisk lagerstyring & \\
    \hline
    \textbf{Aktør} & Depotchefen & \\
    \hline
    \textbf{Præbetingelser} & Et lager med en addresse, penge til at bestille, lager og isbiler er talt op og salgsdata & \\
    \hline
    \textbf{Postbetingelser} & Et lager tilpas fyldt med is & \\
    \hline
    \textbf{Frekvens} & 1 gang om dagen & \\
    \hline
    \textbf{Main Success Scenario} (Flow of events) & \textbf{Aktørhandling} & \textbf{Systemsvar} \\
    \hline
    & 1. Aktøren trykker på "Lager" & 2. Viser lagerbeholdning \\
    \hline
    & 3. Klikker på "Generer lagerplan" &  \\
    & & 4. Planen gemmes og sendes automatisk til plan-fanen \\
    \hline
    & 5. Klikker "Åben plan" & 6. Viser bestillingsplanen med mulighed for at kunne rette i den estimerede bestillingsplan. \\
    \hline
    & 7. Gemmer rettelser ved at klikke "Gem plan" & 8. Opretter bestillinger af varer ud fra bestillingsplanen. \\
    \hline
    & 9. Efter nogle dage registreres de ankomne varer i systemet & 10. Lagerstatus opdateret. \\
    \hline
    \textbf{Alternative flows} & 0.1 Udgåede eller ødelagte varer skal afskrives & \\
    \hline
    & 0.2 Afskrevne varer registreres i systemet. & 0.3 Lager status opdateret. \\
    \hline
    & 4.1 Aktøren vil gerne modificere planen, retter det i plan-fanen og gemmer planen & \\
    \hline
    & & 8.1 Planen er tom. Der kan ikke oprettes en bestilling. Giver en fejlbesked. \\
    \hline
\end{longtable}

\subsection{Kandidattabel}
Nu hvor den centrale Use Case er valgt, kan der nu laves Kandidatklasser. 
Ved opgaven Automatisk lagerstyring skal \textbf{brugeren} kunne se lagerbeholdningen i \textbf{fryseren}. Ud fra \textbf{salgsstatistik} og tilbud kan den optimale mængde \textbf{varer} bestilles hjem. Det skal være muligt at generere en \textbf{lagerplan}, åbne den, og oprette en eller flere \textbf{bestilling} med den.

\begin{longtable}{ |p{120pt}|p{120pt}|p{120pt}| }\label{fig:Kandidatklasser}
    %\hline %why the fuck doesn't this work?
    \textbf{Kandidat Til Klasse} & \textbf{Vurdering} & \textbf{Konklusion} \\
    \hline
    Fryser & Er en fysisk ting der køler is & OUT \\
    \hline
    Sale & Indeholder salgsoplysninger fra et salg ude ved kunden & klasse \\
    \hline
    Product & Indeholder productoplysninger - pris, ID & klasse \\
    \hline
    Customer & Indeholder kundeoplysninger & klasse \\
    \hline
    CustomerOrder & En ordre lavet af en kunde som skal leveres til kunden & klasse \\
    \hline
    DepotOrder & En bestilling af is til depotet & klasse \\
    \hline
    Seller & Indeholder sælgeroplysninger, bliver forbundet med Sale & sale \\
    \hline
    Boss & Indeholder adminstratoroplysninger & klasse \\
    \hline
    DepotPlan & Beskriver hvordan depotet skal ændre sig som sæsonerne gør & klasse \\
    \hline
\end{longtable}
Nu er Kandidatklasserne fundet, og det er nu muligt at opstille en domænemodel af systemet, så basisformen af systemet kan designes, samt forholdene mellem klasserne.
