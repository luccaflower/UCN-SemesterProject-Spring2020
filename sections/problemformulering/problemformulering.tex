\chapter{Problemformulering}\label{ch:problemformulering}

\subsubsection{Problemområde}

Hver isbil er fyldt op med et bestemt antal is-pakker. Når en pakke bliver solgt registrerer systemet det således, at den bestemte is-variant kan blive bestilt til genopfyldning af is-bilerne. Her vil det gerne implementeres, at systemet kan med en algoritme og tidligere salgsdata lave en estimeret bestillingsplan. Dvs. at bestille den optimale mængde af varer ud fra popularitet. 
Efter hver dag er slut skal resultaterne bogføres. Dette bliver gjort ved manuelt at skrive salgsdata over i et Excel ark. Her ønskes at systemet automatisk kan oprette et Excel ark til bogføring som blot skal godkendes.
Projektets formål er at automatisere nogle af de mange Use Cases, som depotchefen allerede udfører, hvilket kan spare dem tid og penge.

\subsubsection{Problemformulering}
 \textit{Hvordan kan vi designe et nyt bestillings- og bogføringssystem til Hjem-IS, som kan automatisere arbejdsgangene og mindske økonomisk tab?}