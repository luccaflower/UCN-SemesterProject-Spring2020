\chapter{Systemvision}\label{ch:systemvision}

\section{Formål og afgrænsing}
Projektets afgrænses til at udvikle et brugervenligt system, som automatiserer bestilling af varer. 
Formålet er at spare firmaet penge og tid på en kompleks opgave.

\section{Problemområde}
Hver isbil er fyldt op med et bestemt antal is-pakker. Når en pakke bliver solgt registrerer systemet det således, at den bestemte is-variant kan blive bestilt til genopfyldning af is-bilerne. Her vil det gerne implementeres, at systemet kan med en algoritme og tidligere salgsdata lave en estimeret bestillingsplan. Dvs. at bestille den optimale mængde af varer ud fra popularitet. 
%Efter hver dag er slut skal resultaterne bogføres. Dette bliver gjort ved manuelt at skrive salgsdata over i et Excel ark. Her ønskes at systemet automatisk kan oprette et Excel ark til bogføring som blot skal godkendes.


\section{Problemformulering}
\textit{Hvordan kan vi designe et nyt lagersystem til Hjem-IS, som kan automatisere bestilling af varer?}

\section{Interessenter og brugere}
Brugeren vil få gavn af at nemmere kunne udføre en af sin opgave, og det vil på længere sigt gavne både kunder, medarbejdere og investorer. Såfremt systemet fungerer optimalt, vil der kunne spare penge og tid 

\section{Teknologi}
Der skal bruges en database, fx en lille server der er forbundet med de andre computerer på arbejdspladsen. Kun brugere med logintilladelse kan bruge systemet. Der skal være en computer på arbejdspladsen, der kan benyttes når man skal ind i databasen. Internet er også et krav for, at en database kan komme op at køre og for at medarbejderne kan bruge nettet til at holde sig opdateret.
