\chapter{Tests}\label{ch:tests}
Med koden designet og implementeret, kan tests af koden påbegyndes. Test er en del af Unified process modellen, hvori den fremgår både under Inception og Elaboration fasen (det gælder også construction og Transistion, men det ikke relevant for dette project). Test er en nødvendig del af Unified process, ikke fordi det program der bliver lavet ikke må have nogle fejl, men fordi det kan give et godt overblik over hvad der kan forbedres\cite{sestoft2008systematic}. Der er brugt JUnit 5 til at teste de forskellige metoder. Herunder kan det ganske vidst ses, at White-Box testing, samt Black-box testing benyttes. 

\section{Fremgangsmåde}
White-box testing er at teste selve programmet. Altså, at alle delene af programmet er blevet udført. Det gode ved White-box testing er, at det både er en god systematisk metode til at opdage fejl, samt også en hurtig og effektiv metode. Måden det foregår på, er at for hver eneste data input sæt der bruges, skal der også være et forventet output specifiseret. Herefter vil programmet køre med alle data input sæt og de givende outputs bliver sammenlignet med de forventede\cite{sestoft2008systematic}.

Black-Box skal sørge for at programmet skal løse det problem den er sat til at løse. Derfor vil det være en god ide, hvis testeren på forehånd har en ide om hvilke problemer det er, for at kunne få programmet til at løse det\cite{sestoft2008systematic}. 

Vores tests bliver primært unit-tests, der er implementeret efter AAA-mønstret (Arrange, Act, Assert), hvor tests bliver eksekveret i tre skridt\cite{ArrangeActAssert}:
\begin{description}
    \item[Arrange:] præ-konditioner for testen sættes op.
    \item[Act:] metoder, der testes på, bliver eksekveret
    \item[Assert:] forventede udfald bliver sammenlignet med faktiske udfald 
\end{description}

\section{Implementering af tests}
Vi gør brug af følgende biblioteker til at køre vores tests:
\begin{itemize}
    \item JUnit 5.4
    \item Mockito 3.3.3
\end{itemize}

\subsection{JUnit}
JUnit er et Java testing framework, der bruges til at definere test-klasser med tilhørende test-metoder. Disse test-metoder instantierer og kalder metoder i en klasse i programmet, kaldet SUT eller "System Under Test". En test-klasse, der gør brug af JUnit, vil typisk følge en struktur lignende eksemplet i Listing~\ref{lst:JUnitEksempel}

I dette eksempel bliver følgende annotationer brugt\cite{JUnitDocumentation}: 
\begin{description}
    \item[\texttt{@BeforeAll}:] Denne metode bliver kaldt før alt andet i klassen.
    \item[\texttt{@BeforeEach}:] Denne metode bliver kaldt før hver enkelt test.
    \item[\texttt{@Test}:] Test-metode.
    \item[\texttt{@AfterEach}:] Bliver kaldt efter hver test.
    \item[\texttt{@AfterAll}:] Bliver kaldt til allersidst.      
\end{description}

Yderligere gøres der brug af JUnit's assert-metoder. En assert-metode definerer et forventet udfald af en test. Der findes adskillige assert-metoder i JUnit, der definerer forskellige former for forventede udfald, herunder \texttt{assertEquals}, \texttt{assertTrue}/\texttt{assertFalse}, og \texttt{assertArrayEquals} som sammenligner variabler med deres forventede værdier. Yderligere findes \texttt{assertNull}/\texttt{assertNotNull}, som tester om en givet reference er null. Til sidst findes \texttt{assertAll} og \texttt{assertThrows}, som bruges til at teste, at givne funktionskald henholdsvis eksekverer uden fejl og kaster en undtagelse\cite{JUnitDocumentation}.

Vi gør brug af JUnit 5.4 grundet kompabilitet med mocking-biblioteket Mockito.

\begin{listing}
    \inputminted[
        frame=lines,
        framesep=2mm,
        baselinestretch=1.2,
        bgcolor=LightGray,
        fontsize=\footnotesize,
        linenos
    ]{java}{listings/junitExample.java}
    \caption{Exempel på en JUnit test-klasse\label{lst:JUnitEksempel}}
\end{listing}

\subsection{Mockito}
Mockito er et mocking-bibliotek, der bruges i samspil med JUnit. Det tillader os at "mocke", det vil sige, lave stedfortræder-klasser i vores tests. Det har det formål, at vi kan isolere vores SUT fra dets dependencies, så vi kan udføre unit-tests på de metoder, der gør brug af disse dependencies. Dette er specielt brugbart når vi tester vores controller-klasser.

I Listing~\ref{lst:MockEkempel} ses det hvordan mocks bliver defineret som felter i en test-klasse. \texttt{@Mock}-annotationen signalerer, at det er en stedfortræder-klasse. \texttt{@InjectMocks} signalerer, at stedfortræderne skal indsættes i denne klasse.

I Listing~\ref{lst:MockTestEksempel} ses et eksempel på en test-metode, der gør brug af mocks. Nøgleordene \texttt{when} og \texttt{verify} er statiske metoder i Mockito-biblioteket. I eksemplet bruges \texttt{when} sådan, at når \texttt{myMethod} is \texttt{MyClass1} bliver kaldt, så skal den returnere '1' \textit{i stedet for} at kalde den reelle metode. 

\begin{listing}
    \begin{minted}
    [
        frame=lines,
        framesep=2mm,
        baselinestretch=1.2,
        bgcolor=LightGray,
        fontsize=\footnotesize,
        linenos
    ]{java}
@Mock private MyClass1 myMockedClass1;
@Mock private MyClass2 myMockedClass2;
@InjectMocks private ClassUnderTest classUnderTest;
    \end{minted}
    \caption{Eksempel på brug af mocks \label{lst:MockEkempel}}
\end{listing}

\begin{listing}
    \begin{minted}
    [
        frame=lines,
        framesep=2mm,
        baselinestretch=1.2,
        bgcolor=LightGray,
        fontsize=\footnotesize,
        linenos
    ]{java}
@Test
public void testMethod() {
        when(myMockedClass1.myMethod()).thenReturn(1);
        classUnderTest.methodUnderTest();
        verify(myMockedClass1, times(1)).myMethod();
        verify(myMockedClass2, never()).otherMethod();
}
    \end{minted}
    \caption{Eksempel på brug af mocks i en test-metode\label{lst:MockTestEksempel}}
\end{listing}

\texttt{verify} fungerer som en assertion: den tester i bund og grund, om en givet metode bliver kaldt, hvor mange gange den bliver kaldt, og hvilke argumenter den bliver kaldt med. I eksemplet bliver det først asserted, at \texttt{myMethod} i \texttt{MyClass1} bliver kaldt én gang med en tom parameterliste. Dernæst bliver det asserted, at \texttt{otherMethod} i \texttt{MyClass2} aldrig bliver kaldt. Dermed kan vi teste på, hvilke beskeder bliver sendt til andre klasser.

\section{Test cases}
For at gå systematisk til værks, skal der opsættes et skema med test cases\cite{Heumann}. Der bliver altså lavet flere scenarier som først blive sat op i skemaet, så testet i programmet, og til sidst vil der foretages ændringer i koden som kan fikse de opståede problemer, hvorefter de samme tests køres igen. Vores test cases, som vi har designet vores tests ud fra, kan ses i Bilag~\ref{app:testcases}.

\subsection{Unit tests}
\subsubsection{Regression Tests}
Test-klassen \texttt{RegressionTest} tester vores regressions algoritme, og for at teste dette skal der bruges en massiv mængde data. Derfor bruges en csv-fil, som beskriver relationerne mellem forskellige typer cement, denne csv-fil kan findes på \texttt{res/concrete.csv}, og selve test koden kan findes på \texttt{src/hjem/is/test/RegressionTest.java}. I disse tests bruges en assert metode, som vi selv har lavet, kaldet \texttt{assertApproxEquals}. Denne bruges eftersom at regression ikke handler om at lave et 100\% perfekt estimat, men bare godt nok til at det er brugbart.

Ved udførsel af testen hentes concrete.csv filen ind i en \texttt{@BeforeAll}, eftersom at alle test bruger den samme data. Data'en laves om til \texttt{Observation} objekter, eftersom at det er den type objekt vores implementation bruger som argument. Begge tests tester at algoritmen kan forudsige at \texttt{compressivestrength} på index 30 passer index for en delta af 20 af den faktiske værdi, begge tests printer også deres $r^{2}$ værdi og deres præcision.

Den første test tester \texttt{GradientDescentFit}\cite{GradientDescent} implementation, hvor der bruges en alpha af 0.0001, den anden tester \texttt{NormalEquationFit}\cite{NormalEquation}. Begge tests består her. 

\subsubsection{Controller tests}
\texttt{StoragePlanControllerTest} og \texttt{PeriodicPlanControllerTest} er unit test-klasser for henholdsvis \texttt{StoragePlanController} og \texttt{PeriodicPlanController}. Disse tests er lavet ud fra vores test cases - vi har kigget på hver enkelt test case og vurderet, hvad den enkelte controller-klasses ansvar er i et givet scenarie, og så har vi skrevet en test, der tester for dét. 

I \texttt{StoragePLanControllerTest} har vi defineret følgende tests:
\begin{itemize}
    \item generatesNewPlanWithNameJuni
    \item generatesNewPlanWithNameJuli
    \item generatesNewPlanWithName100
    \item tellsDatabaseToSaveNewStoragePlan
    \item tellsDataBaseToUpdateExistingStoragePlan
    \item setsCurrentPlanActive
    \item tellsDatabaseToDeletePlanByName
\end{itemize}
Disse tests er navngivet efter det forventede udfald.

I disse tests har vi mocket \texttt{IStoragePlanStore} og \texttt{PeriodicPlanController}, da disse begge indeholder kompleks funktionalitet, som \texttt{StoragePlanController} er afhængig af. Vi mocker ikke model-klasser, da disse kun indeholder data og simpel funktionalitet.

\texttt{generatesNew}-tests, som kan ses i Listing~\ref{lst:generatesNewEksempel}, tester, at en plan med et givet navn bliver oprettet i systemet. Det assertes, at \texttt{spc.get()} returnerer en \texttt{StoragePlan} med det angivne navn, og at der er oprettet en liste, der kan indeholde \texttt{PeriodicPlan}-objekter. 

I \texttt{tellsDatabase}-tests, som ses i Listing~\ref{lst:tellsDatabaseEksempel}, tester vi, om vores controller-klasse sender de korrekte funktionskald til DAO. I setup-fasen sørger vi for, at det \texttt{StoragePlan}-objekt, som er oprettet til testen eksisterer i systemet. Da \texttt{StoragePlanController} ikke har en simpel \texttt{setCurrent()}-metode, bruger vi \texttt{select()} i stedet, som giver besked til DAO om at returnere en \texttt{StoragePlan} med det navn. DAO er i testen en stedfortræder, som er sat til at returnere det objekt, vi bruger i testen. \texttt{select()} setter dernæst det returnerede objekt til \texttt{current}. Betingelsen for, om en plan allerede eksisterer i databasen eller ej er, at \texttt{StoragePlanObjektet}'s \texttt{id}-felt er sat til en værdi. Hvis det er, skal den gemmes som eksisterende, ellers skal der gemmes en ny. Dette tester vi med \texttt{verify}, hvor vi asserter at den korrekte metode i DAO bliver kaldt.

\begin{listing}[h]
    \begin{minted}[
        frame=lines,
        framesep=2mm,
        baselinestretch=1.2,
        bgcolor=LightGray,
        fontsize=\footnotesize,
        linenos,
        breaklines
    ]{java}
@Test
public void generatesNewPlanWithNameJuni() {
    String planName = "Juni";
    spc.generateNew(planName);
    assertEquals(planName,spc.get().getName());
    assertNotNull(spc.get().getPeriodicPlans());
}
    \end{minted}
    \caption{Eksempel på en \texttt{generatesNew}-test\label{lst:generatesNewEksempel}}
\end{listing}

\begin{listing}[h]
    \begin{minted}[
        frame=lines,
        framesep=2mm,
        baselinestretch=1.2,
        bgcolor=LightGray,
        fontsize=\footnotesize,
        linenos,
        breaklines
    ]{java}
@Test
public void tellsDatabaseToSaveNewStoragePlan() {
    StoragePlan sp = new StoragePlan("June", false, new StorageMetaData(100f), new ArrayList<PeriodicPlan>());
    try {
        when(spsMock.getByName("June")).thenReturn(sp);
    } catch (DataAccessException ignored) {
    }
    spc.select("June");
    spc.save();

    try {
        verify(spsMock, times(1)).add(sp);
        verify(spsMock, never()).update(sp);
    } catch (DataAccessException ignored) {
    }
}
    \end{minted}
    \caption{Eksempel på en \texttt{tellsDatabase}-test\label{lst:tellsDatabaseEksempel}}
\end{listing}

\texttt{PeriodicPlanControllerTest} har følgende tests:
\begin{itemize}
    \item setsProductAmountFrom50To40
    \item failsToSetProductAmountToNegativeValue
    \item failsToSetProductAmountToZero
    \item setsProductAmountFrom50To9999
    \item savesPeriodicPlanToDatabase
\end{itemize}
Her ses det, at vi har to former for \texttt{setsProductAmount}-tests. Eksempler på disse findes i Listing~\ref{lst:setProductAmountTests}. I disse test assertes det, at \texttt{amount} af et \texttt{Product}-objekt enten sættes til den angivne værdi eller bliver afvist. Dernæst har vi \texttt{savesPeriodicPlanToDatabase}, som ses i Listing~\ref{lst:savesPeriodicPlanToDatabase}. Her bruges Mockito's \texttt{InOrder}-klasse til at bekræfte, at de \texttt{PeriodicPlan}-objekter, der bliver gemt, er korrekte og i korrekt rækkefølge.

\begin{listing}[h]
    \begin{minted}[
        frame=lines,
        framesep=2mm,
        baselinestretch=1.2,
        bgcolor=LightGray,
        fontsize=\footnotesize,
        linenos,
        breaklines
    ]{java}
@Test
    public void setsProductAmountFrom50To40() {
        ppc.setProductAmount("Isbåd", 40);
        assertEquals(40, ppc.getAmountByName("Isbåd"));
}
    
@Test
public void failsToSetProductAmountToNegativeValue() {
    assertThrows(IllegalArgumentException.class, () -> ppc.setProductAmount("Isbåd", -2));
}
    \end{minted}
    \caption{\texttt{setsProductAmount} og \texttt{failsToSetProductAmount}-tests \label{lst:savesPeriodicPlanToDatabase}}
\end{listing}

\begin{listing}[h]
    \begin{minted}[
        frame=lines,
        framesep=2mm,
        baselinestretch=1.2,
        bgcolor=LightGray,
        fontsize=\footnotesize,
        linenos,
        breaklines
    ]{java}
@Test
public void savesPeriodicPlanToDatabase() {
    InOrder inOrder = inOrder(ppsMock);
    ppc.save();
    try {
        inOrder.verify(ppsMock).update(argThat((PeriodicPlan ppArg) -> ppArg.getProductMap().containsKey(leftProduct)), eq(false));
        inOrder.verify(ppsMock).update(argThat((PeriodicPlan ppArg) -> ppArg.getProductMap().containsKey(product)));
        inOrder.verify(ppsMock).update(argThat((PeriodicPlan ppArg) -> ppArg.getProductMap().containsKey(rightProduct)), eq(false));
    } catch (DataAccessException ignored) {
    }
}
    \end{minted}
    \caption{\texttt{setsProductAmount} og \texttt{failsToSetProductAmount}-tests \label{lst:savesPeriodicPlanToDatabase}}
\end{listing}

Resultater for alle unit tests findes i Figur~\ref{fig:unittestresults}.

\begin{figure}
    \centering
    \includegraphics[scale=1]{figures/tests/unittestresults}
    \caption{Unit test-resultater \label{fig:unittestresults}}
\end{figure}

\subsection{Integration tests}
Integration tests er tests som beskriver og tester hvordan 2 lag interager med hindanden, et godt eksempel er test af DAO objekter, eftersom at både metoden på DAO objektet, og om dens interaktion med databasen bliver testest i denne test case
\todo{write stuff here about our integration tests}
%most important tests
%What it tests
%how it works
%did it pass?

\subsection{System tests}
For at gå endnu et niveau op, blev der også lavet system tests\cite{TestLevels}. Under system testing afprøves forskellige flows af den centrale use case, i dette tilfælde vores Fully Dressed Use Case fra kapitel \ref{fullydressed}.

Under figur \ref{fig:testgenerate} er første iteration af test af UI blevet testest. Herunder det det blevet testet, om det muligt for en lagerplan at blive oprettet. Det viser sig, at det er muligt at oprette lagerplanen med både forskellige navne og tal. Dog bliver den gemte lagerplan ikke vist. Stadig en succes, eftersom målet var at få lagerplanen gemt, herefter kan det implementeres, at den gemte lagerplan også bliver vist frem. På den anden side blev test case nr. 4 ikke en succes, da den gemte navnet "navn", hvilket ikke skulle være meningen. I stedet skulle den foreslå navneændring.

\begin{figure}[p]
    \centering
    \includegraphics[width=0.7\hsize]{figures/tests/test_generer_plan.png}
    \caption{Scenarie: En lagerplan skal genereres}
    \label{fig:testgenerate}
\end{figure}

Det skulle herunder gerne være muligt, at rette en eksisterende lagerplan. Testskemaet på figur \ref{fig:testedit} tester, at når man er inde på lagerplanen, skulle det være muligt at redigere mængden af varer, om planen er aktiv eller ej, og hvilken periode man redigerer.  Dette er dog ikke muligt at gøre, når display af lagerplanen ikke eksisterer. Heri blivet test evalueret og derefter skal det impelmeteres således, at det muligt at klikke ind på en lagerplan og rette den. 

\begin{figure}[p]
    \centering
    \includegraphics[width=0.7\hsize]{figures/tests/edit_existing_plan.png}
    \caption{Scenarie: En lagerplan skal rettes}
    \label{fig:testedit}
\end{figure}

Det skulle være muligt at generere en plan, rette i planen ved fx at tilføje et produkt, og så gemme planen igen. Ud fra testskemaet på figur \ref{fig:testeditperiod} kan det ses, at til trods for at gem-metoderne er testet, så fungerede det ikke samme med UI på dette tidspunkt. Det skal evalueres sådan at det kan virke ordenligt.

\begin{figure}[p]
    \centering
    \includegraphics[width=0.7\hsize]{figures/tests/test_ret_plan.png}
    \caption{Scenarie: En periode i lagerplanen rettes}
    \label{fig:testeditperiod}
\end{figure}

Det skulle være muligt at slette en lagerplan, som allerede er oprettet. Som testskemaet på figur \ref{fig:testdelete} viser, så var det på det tidspunkt ikke muligt at vise planerne så man kunne tilgå "slet" knappen. Derfor fejler begge tests. For at løse det skulle planerne selvfølgelig vises, sådan det er muligt at rette og slette dem. UI for lagerplanen var eller implementeret og virkede, men under testen her fejlede det.

\begin{figure}[p]
    \centering
    \includegraphics[width=0.7\hsize]{figures/tests/test_slet_lagerplan.png}
    \caption{Scenarie: En lagerplan skal slettes}
    \label{fig:testdelete}
\end{figure}





