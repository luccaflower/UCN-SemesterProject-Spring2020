\chapter{Tests}\label{ch:tests}
Nu hvor koden er designet implementationen er forklaret, kan tests af koden påbegyndes. Testning er en del af Unified Process modellen, hvori den fremgår både under Inception og Elaboration fasen (det gælder også construction og Transistion, men det ikke relevant for dette project)\cite{UP}. Test er en nødvendig del af Unified process, ikke fordi programmet ikke må have nogle fejl, men fordi det kan giver et godt overblik over hvilke fejl der opstår i programmet, så det senere kan debugges og forbedres\cite{sestoft2008systematic}. Herunder kan det ses, at White-Box testing, samt Black-box testing benyttes. 

White-box testing er at teste selve programmet. Altså, at alle delene af programmet er blevet udført. Det gode ved White-box testing er, at det både er en god systematisk metode til at opdage fejl, samt også en hurtig og effektiv metode. Måden det foregår på, er at for hver eneste data input set der bruges, skal der også være et forventet output specifiseret. Herefter vil programmet køre med alle data input set og de givende outputs bliver sammenlignet med de forventede\cite{sestoft2008systematic}.

Black-Box skal sørge for at programmet skal løse det problem, den er sat til at løse. Derfor vil det være en god ide, hvis testeren på forehånd har en ide om hvad et muligt problem kunne være, for at kunne få programmet til at løse det\cite{sestoft2008systematic}. 
