\chapter{Konklusion}\label{ch:konklusion}

Ud fra interviewet med Hjem-IS, lykkedes det at finde et problem, som omhandlende ikke-optimal brug af tid. Til dette problem blev der ved brug af Unified Process\cite{UnifiedProcess} udviklet en løsning i form af et automatisk lagerstyringssystem, som kunne automatisere en af de centrale arbejdsgange og spare firmaet for tid og fejl. Løsningen brugte regression til at lave gode modeller for hvor meget og hvornår det er optimalt at bestille varer. Systemet var baseret på en trelagsarkitektur med database og brugergrænseflade, og skulle ende ud i et program som funktionelt kunne løse problemet.
Det lykkedes næsten at få lagersystemet til at fungere som det skulle, men der opstod udfordringer under udviklingen af UI, som gav problemer. Til trods for at funktionaliteten af programmet virkede, så var UI det største problem. 

Programmet skal videreudvikles sådan at UI virker, og løsningen bliver mere kompetent. De tests der blev foretaget vil blive gentaget efter programmet er blevet fixet, og vil forhåbentlig virke som det skal derefter. Udover dette ville det også være muligt senere at implementere flere use cases fra tidligere i rapporten, såsom automatisk bogføring, sådan at programmet kan give endnu mere værdi for Hjem-IS.

Unified Process har været nyttigt til at give et overblik over, hvilke aktiviteter og faser man skal igennem for at kunne gå fra interview, til problemstilling, til løsning i form af et system. Unified Process har hjulpet os med ikke at komme ud af kurs, lave vurderinger ud fra risici, samt at gruppen under hele processen har foretaget større overvejelser til aktiviteterne under de 2 iterationer vi har været igennem, for at en rapport over arbejdet og beslutninger taget undervejs kan udarbejdes. De forskellige discipliner der indgår i Unified Process har hjulpet med at forstå hvilke trin, der indgår i at gå fra abstrakt idé til færdigt program, og hver disciplin har gjort, at der er taget vigtige beslutninger om, hvad løsningen skulle være under hver disciplin. 

%Vi at interviewe Hjem-IS og analyserede os frem til hvilken kultur der er hos Hjem-IS, samt hvilke arbejdsgange der er. Produktet af Inception fasen var en business case, som viste at der var flere centrale opgaver der kunne optimeres. Ud fra forundersøgelsen, hvor arbejdsgangene blev udpenslet, blev der også udviklet mockups af, hvordan et system til at håndtere de centrale Use Cases kunne se ud.
%Ud fra forundersøgelsen kunne der laves et systemvision, hvor hovedmålet med dette projekt blev præciseret til at være, at oprette et automatisk lagertyringssystem til Hjem-IS. Hvilke interessenter, fordele og teknologier der er nødvendige for at kunne gennemføre et sådan system, blev også præciseret. Dette blev konkluderet på baggrund af business casen, samt forundersøgelsen. 
%Her endte første iteration under Inception fasen, hvorefter Elaboration fasen begyndte. Eftersom projektet har fokuseret ind på én bestemt Use Case, under forundersøgelsen, begyndte gruppen at undersøge hvilke nødvendige krav, der kunne få et Automatisk Lagerstyring system til at fungere. 
%Under Krav aktiviteten blev funktionelle- og informationskravene for sådan et system diskuteret. Her blev en Fully Dressed Use Case udviklet, der har hjulpet med at præcisere hvilke opgaver, der indgår i Automatisk Lagerstyring. På baggrund af Fully Dressed Use Casen blev der opstillet domænemodeller og relationel modeller, hvor der ud fra risiko vurdering blev præciseret endnu mere, sådan at nogle klasser blev valgt fra.
%Ud fra kravene kunne Analyse og Design aktiviteterne udføres. Her blev der udvilket SSD og Operationskontrakt, som i flere detaljer beskrev, hvordan systemet skulle virke, samt hvad metoderne og klasserne skulle hedde. Med et Kommunikationsdiagram og Designklassediagram blev samtlige klasser, relationer, metodekald, mm. defineret. Herunder blev der anvendt kendte designmønstre, bl.a. DAO, 3-lags-arkitektur og Singleton. 
%Under Implementation aktiviteten blev systemet udviklet i Java, og havde fokus på at implementerer de funktionelle centrale metoder for Use Casen. Udover dette blev der også lavet SQL queries til at snakke sammen med en database, og parallelitet blev anvendt for at optimere systemet. Det lykkedes at få regressions algoritmerne til at fungere, desværre ikke sammen med resten af systemet.
%Løsningen endte med at blive et system med en GUI baseret på mockups, som skulle kunne løse det centrale problem. Under Test aktiviteten blev systemet testet på flere niveauer, og testene viste at funktionelt fungerer systemet, men at GUI ikke fungerer ordentligt sammen med resten af det underlæggende system. 
%Alt i alt lykkedes det at udvikle et system som ud fra tests kan udføre mange af de nødvendige opgaver i Use Casen, men at brugergrænsefladen skal fixes under næste iteration.


