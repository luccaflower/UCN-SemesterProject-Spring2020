\chapter{Introduktion}\label{ch:introduction}
På is-markedet er der konkurrence, og det kan blandt andet mærkes hos Hjem-IS. I denne rapport er Hjem-IS Aalborg depotet blevet interviewet, for at finde et problem, som kunne løses med et nyt IT-system.  
Hos Hjem-IS kunne flere opgaver automatiseres, og det blev valgt at udvikle et automatisk lagerstyringssystem. For at kunne udvikle sådan et system ville det kræve en database med samtidighed, og et program som kan håndtere data i databasen, og opereres med en brugergrænseflade. 
Det skulle så være muligt at lave optimale bestillinger af is, uden at bruge meget tid på selv at beregne gode bestillingsmængder. Er det derfor muligt at designe et nyt lagersystem til Hjem-IS, som kan automatisere bestilling af varer? Og i så fald i optimale mængder? 
Projektet forsøger at besvarer disse spørgsmål og udvikle en løsning, som kan leve op til de krav, der vil være til sådan et system. For at kunne udvikle en løsning bruges Unified Process (UP)\cite{UnifiedProcess} som udvilkingsmetode. Den beskæftiger sig med en Use Case og risiko drevet udviklingsprocess. UP sikrer at der ikke bliver brugt for meget tid på irrelevante ting tidligt i projektet, hvilket betyder at fejl og rettelser findes tidligt i projektet. De forskellige discipliner under hver fase af UP vil blive gennemgået sådan, at hele processen forekommer løbende i rapporten. Da UP er en længere iterativ process, vil der kun være tid til Inception fasen og første Iteration af Elaboration fasen\cite{UnifiedProcess}.