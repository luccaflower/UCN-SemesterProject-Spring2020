\chapter{Business case}\label{ch:businesscase}
Hjem-IS er en af de få specialforretninger tilbage i Danmark
Virksomheden er grundlagt på, at sælgere kører ud på de forskellige ruter tildelt. Hjem-Is kom til Danmark i 1976, efter at svenskeren Eric Ericsson først bragte ideen til livs i sverige ved at spænde en fryseboks bag på cyklen.
Her ringes med klokken, for at informere folk om at is-bilen er til stede. For at kunne holde virksomheden igang kræver det god planlægning ift. hvor man skal køre og ringe med klokken, hvilke produkter man skal have på lager og hvor mange, og samtidig kunne håndtere specialordre. Der er meget planlægning og informationer der skal holdes styr på, og optimeres så alle kunderne kan få deres IS. Det betyder at der kræves pålideligt styringssoftware til at håndtere disse opgaver. 